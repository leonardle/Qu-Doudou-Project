\chapter*{前言}
\addcontentsline{toc}{chapter}{前言}

近代数学的一个特点是公理化、结构化。
代表近代数学的布尔巴基(Bourbaki)
\index{Bourbaki\kong 布尔巴基(学派)}
学派认为“数学是研究抽象结构的学科”。
本讲义力图介绍近代数学中的四大基本结构:
序结构、代数结构、拓扑结构、测度结构,
从而勾勒出近代数学基本框架,
满足那些“想要多学一些数学”的爱好者们的基本需求,
并为其进一步学习近代数学奠定基础。

虽说声称是“零基础入门高等数学”,
本讲义其实并不是真正的“白手起家”,
还是要假定读者对一些基本的数学对象有直观的了解,
包括:实数的加减乘除运算、实数大小的比较,仅此而已。

事实上,近代数学基础完整的构建顺序大概是:
命题逻辑$\rightarrow$
一阶谓词逻辑$\rightarrow$公理集合论
$\rightarrow$自然数公理系统
$\rightarrow$实数理论。什么是自然数中的“0”,什么是“1”,
“自然数”是在集合论的框架下定义出来的,并不是“本来就有的”;
而自然数的加法运算也是定义出来的,
不再是初等数学那样子来源于日常生活经验,理所当然的。
这也是为什么罗素用了300多页才得到了自然数“1”的定义,
这才是真正的“白手起家”。
有了自然数,之后再定义整数,
通过对整数环分式化来构造有理数,
再对有理数完备化来构造实数。

而本讲义跳过这个漫长的构建过程
(但会在正文各处穿插地简要提一下),
直接假定读者知道什么是实数,
以及实数的加减乘除、大小比较。
这样做的原因有两点,其一是避免讲义篇幅过长,
过分纠结于数学基础是枯燥乏味的,
毕竟这个讲义不是专门的“数学基础”教材。
其二是为我们要介绍的抽象结构提供具体例子,
比如讲到“偏序集”这个结构时,
实数配以通常的比较大小就是一个例子;
讲到“群”结构时,实数配以加法运算就是一个最基本的例子。
离开具体例子空谈抽象概念并不利于对数学概念的理解
(当然格罗滕迪克Grothendieck是例外)。
\index{Grothendieck\kong 格罗滕迪克}

由于这个讲义是笔者写着玩的,
且笔者水平有限,胡说八道之处在所难免。
希望读者批判性地看待本讲义中的某些论断。
另外由于匆忙成稿,笔误、语法错误应该也挺多的,
恳请批评指正。

\section*{关于数学学习:给初学者}

如果你是初学者,请忘掉中学所学;

如果之前接触过“高数”,也请忘掉。

中学数学讲究“知识点”,重在“刷题、熟练”,
强调解题技巧——这些东西在本讲义中不再重要。

近代数学不再是“一个个的知识点”,
而是知识体系的摩天大厦。
学习近代数学,不仅仅是能记住多少公式定理、
能解决多少问题(当然解决问题很重要,
比如微积分诞生之时就威力无比所向披靡),
更多的是将学到的知识有条理地组织起来,
形成理论体系。

于是,概念、定理证明就格外重要。
话不多说,建议初学者学习本讲义时,
“不动笔墨不读书”,“抄书三遍,其义自见”。
尤其是读到\textbf{定义}、\textbf{定理的证明}时,
有时只用眼睛看是没有效果的,这时候请反复诵读、抄写
(看似笨的办法其实是捷径),直到能脱稿复述为止;
等功底到了一定程度,可以先用眼睛看定理的证明过程,
多看几遍明白大致思路,合上书,在纸上复述、默写;
最高境界是,读完定理所陈述的内容后,
合上书,不去看定理的证明过程,
自己思考、补全证明(许多大数学家的境界如此)。
学习近代数学某种意义下更像学外语,贵在反复、坚持。
抄写、默写、背诵、复述是初学者的常态。
没有扎实的基础,何谈今后灵活运用。\vs

第一章数理逻辑比较简单,
仅仅是将我们众所周知的推理规则
用符号语言重新梳理了一遍,并没有涉及深入的内容。

从第二章集合论开始,“抄书”就格外重要了。
有一些概念可能对初学者来说过于抽象,不知所云
(尤其是第四章代数结构,这将是重灾区);
或者是一些本来我们都明白、觉得理所当然的概念,
被叙述得面目全非——这时候请相信笔者,
先反复抄书强迫自己记住,日后再反复体会。
这就像小时候背过的古诗,
等到长大后的不经意间才突然明白诗中意境。

在第五章实数的完备性,
我们将第一次见到稍具规模的理论体系——
若干个证明较复杂的定理被组织成一套理论。
在此,“抄书大法”将大有可为。\vs

学习数学贵在坚持,毅力比智商更重要。
只要能坚持住,本讲义的全部内容人人都能学会。

苍蝇飞行灵敏、反应迅捷,
但是永远无法像苍鹰般翱翔天空;
学数学也不在于反应快、小聪明(可以搞金融呀),
毕竟我们的征途是星辰大海,
我们面对的是摩天大厦。

\section*{本讲义各章内容安排}

(待补)

\newpage

\section*{致谢}

感谢笔者母校中国科学技术大学数学系的培养

感谢豆瓣小组提供平台

感谢一起编纂此讲义的小伙伴们
\vspp

\subsection*{贡献者(豆瓣昵称,按拼音字母顺序排序)}

曲豆豆,王有为,在云上,子坚,
Pika*Pie

(待补)

\newpage

\begin{centering}
\begin{Huge}
\textbf{《圣经:创世纪》节选}
\end{Huge}\vsp
\large

起初,神创造逻辑。

逻辑是空虚的。

神的零运行在水面上,

这是平凡的。

神说,要有集合。

于是就有了集合。

神看集合是好的,

就将其公理化。

有集合,有运算,

就是一个代数结构。\vsp

神照自己的样子造人,

于是就有了0.

神在东方的伊甸建立了一个园子,

称之为群。

神将0安置在那里,

让它作幺元。

起初,伊甸园是平凡的。

神说,园中所有果子你可以随便吃。

只有后继树上果子,你不能吃。

因为你吃的时候必死。\vsp

神说,那人独居不好。

于是要为它造一个配偶。

神让0沉睡,取它的后继数1.

神把1带到伊甸园,

让它和0配对,

二人从此一起生活。

神教给它们加法的运算,

让它们可以互相交合。

0依然是伊甸园的单位元。

0的逆元是0,1的逆元是1,

0和1彼此交合则又得到1.\vsp

神所造的,

惟有皮亚诺比一切的活物更狡猾。

皮亚诺对1说,

神岂是真说不许你们吃园中所有树上的果子么?

1对皮亚诺说,

园中树上的果子我们可以吃,

唯有后继树上的果子不可,

免得我们死。

皮亚诺说,你们不会死。

神是怕你们吃了后继树上的果子,

能自己创造新的数。\vsp

皮亚诺告诉1它自己是怎么来的,

1对单调的只有0和1的加法已经厌烦。

它看后继树上的果子可做食物,

而且悦人眼目,是可喜爱的,

就摘下果子来吃了。\vsp

于是1对自己取了后继数,是为2.

1又摘下果子给2,2也吃了。

于是2对自己取了后继数,是为3.

很快,后继树下生成了所有正整数。。。

\end{centering}






