\mchapter{朴素集合论}

\section{集合的概念与基本例子(待改)}

\begin{definition}
集合的概念不可言说,不言自明,无须定义。谓词$\in$(读作“属于”)也不可言说。两个集合的相等,也无需定义。
\end{definition}
粗俗地说,“几乎所有你能想到的”的事物、对象,都是集合。
而两个集合相等,通俗地说就是它们是同一个事物。
对于两个集合$A,B$,我们可以构造出命题$A\in B$.当$A\in B$为真命题时,我们称
$A$是$B$的元素。

为避免集合论悖论,我们简单粗暴地避开谈论$A\in A$这个命题。我们不认为一个集合是它自身的元素。
例如,所有集合构成的全体,我们不认为是集合。事实上,并不是随便一些东西放在一起都构成一个集合。
哪些东西放在一起构成集合,哪些不认为是集合,是被集合论公理体系规定的,本讲义不打算深究。
本章最末会对集合论公理做简单介绍。\vs

由所有满足性质$P$的对象$x$的全体构成的集合,记作
$$\{x|P(x)\}$$

当然,对于某些集合,我们也可以通过列举其中的元素来表示这个集合,例如由$1,2,3$
这三个元素构成的集合可以记为$\{1,2,3\}$.

关于集合相等,首先有一条基本的公理:\vs

\textbf{外延公理}:\emph{对于两集合$A,B$,$A$与$B$相等当且仅当它们拥有相同的元素}。

用符号语言表达为:对于任何集合$A,B$,
$$(A=B)\Leftrightarrow[\forall x,(x\in A)\Leftrightarrow(x\in B)]$$

\begin{example}[常见的集合及其记号]
$$\mathbb{Z}:=\{...,-1,0,1,2,3...\}$$
$$\mathbb{Z_+}:=\{x\in\mathbb{Z}|x>0\}$$
$$\mathbb{Q}:=\{\frac{m}{n}|m,n\in\mathbb{Z},n\neq0\}$$
分别是我们熟悉的整数集、正整数集、有理数集。
\end{example}
我们用$\mathbb{R}$来表示实数集,
$$\mathbb{C}:=\{a+b\sqrt{-1}|a,b\in\mathbb{R}\}$$
为复数集。

\begin{example}对于正整数$n$,我们也习惯记
$$n\mathbb{Z}:=\{nk|k\in\mathbb{Z}\}$$
即,由$n$的倍数构成的集合。
\end{example}

我们也习惯用
$$\frac{1}{2}\mathbb{Z}:=\{\frac{n}{2}|n\in\mathbb{Z}\}$$
来表示半整数集。读者可举一反三。

在此假定读者以上集合有基本了解。

\begin{definition}[空集]
定义集合$\varnothing$为
$$\varnothing:=\{x|x\neq x\}$$
称此集合为空集。
\end{definition}
由此可见,空集就是不含任何元素的集合。

\begin{example}
考虑集合$\{\varnothing\}$,注意这是由“空集”这个元素构成的集合,它并不是空集。并且有
$$\varnothing\in\{\varnothing\}$$
\end{example}

事实上,对任何集合$A$,我们都可以考虑由$A$这一个元素构成的集合$\{A\}$,成立$A\in\{A\}$.

\begin{definition}[子集]
对于集合$A,B$,称$A$是$B$的子集,若
$$\forall x\in A, x\in B.$$
此时记作$A\subseteq B$,读作“$A$包含于$B$”,或者“$B$包含$A$”.
\end{definition}
也就是说,“$A$中的所有元素都在$B$中”。可见,对任何集合$A$,都成立$A\subseteq A$.

回顾上一节讲到的逻辑连接词“蕴含”,
$A\subseteq B$的定义也可以写作
$$\forall x, (x\in A)\Rightarrow(x\in B)$$

\begin{prop}
对于任何集合$A$,都有$\varnothing\subseteq A$.也就是说,空集是任何集合的子集。
\end{prop}
\begin{proof}
对于任何集合$x$,按照子集的定义,我们需要证明
$$(x\in\varnothing)\Rightarrow(x\in A)$$
而由空集的定义,对于任何的$x$,$x\in\varnothing$总是假命题,从而
$(x\in\varnothing)\Rightarrow(x\in A)$一定是真命题。得证。
\end{proof}
这个例子可帮助初学者来理解逻辑连接词“$\Rightarrow$”的含义。

\begin{example}
$$\mathbb{Z}\subseteq\mathbb{Q}\subseteq\mathbb{R}\subseteq\mathbb{C}$$
$$\varnothing\subseteq\{\varnothing\}$$
\end{example}

\begin{prop}[集合相等的一个充要条件]
对于集合$A,B$,则$A=B$当且仅当
$$(A\subseteq B)\wedge(B\subseteq A)$$
\end{prop}
\begin{proof}
这是外延公理的一个近乎无聊的应用。读者自行补全细节。
\end{proof}

\begin{definition}[真子集]
对于集合$A,B$,称$A$是$B$的真子集,如果
$$A\subseteq B\text{且}A\neq B$$
此时记作$A\subsetneqq B$,也读作“$A$真包含于$B$”.
\end{definition}

\begin{example}[开区间与闭区间]
再回顾以下我们早已熟悉的实数集$\mathbb{R}$的子集:对于实数$a<b$,有
$$(a,b):=\{x\in\mathbb{R}|a<x<b\}$$
$$(a,b]:=\{x\in\mathbb{R}|a<x\leq b\}$$
$$[a,b):=\{x\in\mathbb{R}|a\leq x<b\}$$
$$[a,b]:=\{x\in\mathbb{R}|a\leq x\leq b\}$$
$$(a,+\infty):=\{x\in\mathbb{R}|x>a\}$$
$$[a,+\infty):=\{x\in\mathbb{R}|x\geq a\}$$
$$(-\infty,a):=\{x\in\mathbb{R}|x<a\}$$
$$(-\infty,a]:=\{x\in\mathbb{R}|x\leq a\}$$
\end{example}

\begin{definition}[有限集与无限集]
对于集合$A$,如果$A$中只有有限多个元素,则称$A$为有限集。反之称为无限集。
\end{definition}
这个概念是直接易懂的。常见的许多集合,
诸如$\mathbb{Z},\mathbb{Q},\mathbb{R},\mathbb{C}$等等,都是无限集。

比较显然的一点是,对于集合$A$,$A$是无限集当且仅当对任何正整数$n$,
存在$A$的一个含有$n$个元素的子集。

\fengexian
 

\section{集合的基本运算(待改)}

本节介绍集合的运算,如何由我们已有的集合来构造新的集合。
\begin{definition}[集合的交、并]
对于集合$A,B$,我们定义
$$A\cap B:=\{x|(x\in A)\wedge(x\in B)\}$$
$$A\cup B:=\{x|(x\in A)\vee(x\in B)\}$$
分别成为集合$A,B$的交集与并集。
\end{definition}

\begin{definition}[补集]
给定集合$X$,对于$X$的子集$A$,定义$A$在$X$中的补集
$A^c:=\{x\in X|x\not\in A\}$
\end{definition}
在谈论补集时,我们总是事先给定集合$X$.

\begin{definition}[差集]
对于集合$A,B$,定义
$$A-B:=\{x\in A|x\not\in B\}$$
\end{definition}

\begin{prop}[交、并的分配律]\label{set-fenpeilv}
对于任意集合$A,B$,成立
$$(A\cup B)\cap C=(A\cap C)\cup(B\cap C)$$
$$(A\cap B)\cup C=(A\cup C)\cap(B\cup C)$$
\end{prop}
\begin{proof}
留作习题。
\end{proof}

\begin{prop}[摩根律]\label{set-morgen}
设集合$A,B$均为集合$X$的子集,我们在$X$中谈论补集。则成立
$$(A\cap B)^c=A^c\cup B^c$$
$$(A\cup B)^c=A^c\cap B^c$$
\end{prop}
\begin{proof}
留作习题。
\end{proof}

\begin{definition}[幂集]
对于集合$A$,定义集合
$$2^A:=\{B|B\subseteq A\}$$
称其为集合$A$的幂集。
\end{definition}
也就是说,$2^A$被定义为由$A$的全体子集构成的集合。

\begin{example}对于集合$A=\{1,2,3\}$,$B=\varnothing$,则成立
$$2^A=\{\varnothing,\{1\},\{2\},\{3\},
\{1,2\},\{1,3\},\{2,3\},\{1,2,3\}\}$$
$$2^B=\{\varnothing\}$$
$$2^{2^B}=\{\varnothing,\{\varnothing\}\}$$
\end{example}

\begin{definition}[有序对]
对于集合$A,B$,定义集合$(A,B)$为
$$(A,B):=\{\{A\},\{A,B\}\}$$
\end{definition}
可见无论$A,B$是什么样的集合,当$A\neq B$时,集合$(A,B)$总是由两个元素构成:
$$\{A\}\in(A,B)$$
$$\{A,B\}\in(A,B)$$
再注意,$\{A,B\}$表示的是“以$A$,$B$这两个元素构成的集合”。

特别地,$(A,A)=\{\{A\},\{A,A\}\}=\{\{A\},\{A\}\}=\{\{A\}\}$.

细心的读者可能会发现符号歧义。对于实数$a<b$,
$(a,b)$可以是开区间$\{x\in\mathbb{R}|a<x<b\}$,
也可以是这里讲的有序对。读者遇到此情况,不妨靠语境来判断它的含义。


\begin{definition}[笛卡尔积]
对于非空集合$A,B$,我们定义
$$A\times B:=\{(a,b)|a\in A, b\in B\}$$
\end{definition}
这里的$(a,b)=\{\{a\},\{a,b\}\}$是有序对。

\begin{example}对于$A=\{1,2,3\}$,$B=\{4,5\}$,则有
$$A\times B=\{(1,4),(1,5),(2,4),(2,5),(3,4),(3,5)\}$$
\end{example}

\begin{example}[$n$维空间]
回顾$\mathbb{R}$是实数集(直线),则有
$$\mathbb{R}^2:=\mathbb{R}\times\mathbb{R}
=\{(x,y)|x,y\in\mathbb{R}\}$$
一般地,对于正整数$n$,定义
$$\mathbb{R}^n:=\mathbb{R}\times...\times\mathbb{R}
=\{(x_1,x_2,...,x_n)|x_i\in\mathbb{R},\forall 1\leq i\leq n\}$$
\end{example}
注意,谈论到多个集合作笛卡尔积时,比如对于$A,B,C$三个集合,原则上应该有

$$(A\times B)\times C=\{((a,b),c)|a\in A,b\in B, c\in C\}$$
$$A\times (B\times C)=\{(a,(b,c))|a\in A,b\in B, c\in C\}$$

讲道理(严格按照有序对的定义),$((a,b),c)$与$(a,(b,c))$通常是不相同的。
本讲义中尽量避免如此繁琐的讨论,暂且认为

$$A\times B\times C:=\{(a,b,c)|a\in A,b\in B, c\in C\}$$

读者可以尝试给“多元有序组”下定义,或者暂且忍受,直到学到集合的映射。\vs

本节最后,简单介绍一下本讲义开篇“圣经”中提到的“后继树(数)”。
\begin{definition}[后继]
对于集合$A$,定义集合
$$A^+:=A\cup\{A\}$$
称为集合$A$的后继。
\end{definition}

例如,对于集合$A=\{a,b\}$,则有
$$A^+=\{a,b\}\cup\{\{a,b\}\}=\{a,b,\{a,b\}\}$$

正如“圣经”中所说,通过不断取后继,可以得到所有的正整数。
具体地,我们将“$0$”定义为空集$\varnothing$,即$0:=\varnothing$.
之后我们令$1:=0^+=\varnothing\cup\{\varnothing\}=\{\varnothing\}=\{0\}$.
再之后,我们将$2$定义为$1$的后继,即
$$2:=1^+=1\cup\{1\}=\{0\}\cup\{1\}=\{0,1\}$$
不断地做下去,有
$$3:=2^+=2\cup\{2\}=\{0,1\}\cup\{2\}=\{0,1,2\}$$
$$4:=3^+=3\cup\{3\}=\{0,1,2\}\cup\{3\}=\{0,1,2,3\}$$
$$......$$
这正是皮亚诺的自然数公理化构造。
在此观点下,0和正整数统称为自然数。

我们不再继续深究自然数公理。

\fengexian

\begin{prob}验证性质\ref{set-fenpeilv}与性质\ref{set-morgen}.
\end{prob}\vs

\begin{prob}对于集合$A,B$,证明以下三个命题是互相等价的:

(1)$A\subseteq B$

(2)$A\cap B=A$

(3)$A\cup B=B$

(4)$A-B=\varnothing$
\end{prob}\vs

\begin{prob}对于集合$A,B,C$,证明:

(1) $(A\subseteq C)\wedge(B\subseteq C)\Leftrightarrow A\cup B\subseteq C$

(2) $(C\subseteq A)\wedge(C\subseteq C)\Leftrightarrow C\subseteq A\cap B$
\end{prob}\vs

\begin{prob}[元素的个数]
设集合$A$为有限集(即,只含有有限多个元素),我们用$|A|$来表示集合$A$中元素的个数。
例如当$A=\{2,7,8\}$时,有$|A|=3$.现在,假设$A,B$都是有限集,证明以下结论:

(1)$|2^A|=2^{|A|}$

(2)$|A\times B|=|A|\times|B|$

(3)$|A\cup B|=|A|+|B|-|A\cap B|$
\end{prob}\vs

\begin{prob}对于非空集合$X,Y,X',Y'$,证明:

(1) $X\times Y\subseteq X'\times Y' \Leftrightarrow (X\subseteq X')\wedge(Y\subseteq Y')$

(2) $(X\times Y)\cup(X'\times Y)=(X\cup X')\times Y$

(3) $X\times Y\cap X'\times Y'=(X\cap X')\times(Y\cap Y')$
\end{prob}\vs

\begin{prob} 对于集合$A,B$,

(1)证明: $(A\subseteq B)\Leftrightarrow (2^A\subseteq 2^B)$

(2)当$A,B$非空时,是否一定成立$2^{A\times B}=2^A\times2^B$?

(2)直接写出集合$2^{\{\varnothing,\{\varnothing\}\}}$的所有元素。

\end{prob}\vs

\begin{prob}对于集合$A,B$,

(1)何时成立$\{A\cup B\}=\{A\}\cup\{B\}$?

(2)何时成立$(A,B)=(B,A)$?
\end{prob}\vs

\section{集合列与集合族(待补)}
本节我们继续介绍集合运算。
\begin{definition}[集合族]
对于集合$\mathcal{A}$,如果$\mathcal{A}$是由一些集合构成的集合,则称$\mathcal{A}$是一个集合族。
\end{definition}
由定义可知,集合族首先是一个集合,只不过这个集合是由一些集合构成的。
例如,集合$\{\varnothing\}$就是一个集合族。对任何集合$A$,$A$的幂集$2^A$也是一个集合族。

其实,“集合族”是一个很无聊的概念,因为任何集合事实上都是集合族,
集合与集合族是一回事。对于这个论断,读者可以承认之,也可以无视之。

\begin{definition}[任意交与任意并]
对于非空的集合族$\mathcal{A}$,我们定义新的集合
$$\bigcap \mathcal{A}:=\{x|\forall A\in \mathcal{A},x\in A\}$$
$$\bigcup \mathcal{A}:=\{x|\exists A\in \mathcal{A},x\in A\}$$
分别称为集合族$\mathcal{A}$的交、并。
\end{definition}
例如,对于$\mathcal{A}=\{\{a,b\},\{b,c\}\}$,则有
$$\bigcap \mathcal{A}=\{a,b\}\cap\{b,c\}=\{b\}$$
$$\bigcup \mathcal{A}=\{a,b\}\cup\{b,c\}=\{a,b,c\}$$
我们更感兴趣的是当$\mathcal{A}$时无限集的时候,此时就会出现无穷多个集合的交(并)。\vs

对于集合族$\mathcal{A}$,我们将$\mathcal{A}$中的每个元素赋以不同的标记来区分:
$$\mathcal{A}=\{A_i|i\in\mathcal{I}\}$$
其中,$\mathcal{I}$称为集合族$\mathcal{A}$的一个\textbf{指标集},它可以是有限集,也可以是无限集。

此时,我们习惯使用记号
$$\bigcap_{i\in\mathcal{I}}A_i\,\,\,\,\,\,\,\,\bigcup_{i\in\mathcal{I}}A_i$$
分别来表示集合族$\mathcal{A}$的交、并。

特别地,当集合族$\mathcal{A}$的指标集$\mathcal{I}=\mathbb{Z_+}$为正整数集时,即
$$\mathcal{A}=\{A_i|i\in\mathbb{Z}_+\}=\{A_1,A_2,A_3,...\}$$
我们更喜欢将集合族$\mathcal{A}$的交、并记作
$$\bigcap_{i=1}^{\infty}A_i:=\bigcap\mathcal{A}$$
$$\bigcup_{i=1}^{\infty}A_i:=\bigcup\mathcal{A}$$

\begin{example}
对于每一个正整数$n$,令集合$A_n:=(-n,n)\subseteq\mathbb{R}$为开区间。则成立
$$\bigcap_{n=1}^\infty A_n=(-1,1)$$
$$\bigcup_{n=1}^\infty A_n=\mathbb{R}$$
\end{example}
\begin{proof}
(待补)
\end{proof}

\section{映射与函数(待补)}
\section{集合的势(待补)}
\section{可数集(待补)}
\section{选择公理与集合论ZFC公理体系简介(待补)}
\section{附加习题}

\begin{prob}[集合的对称差]

对于两个集合$A,B$定义一种新的集合运算$\triangle$,称为对称差,如下:
$$A\triangle B:=(A-B)\cup(B-A)$$
证明对称差运算满足以下性质:

(1) $(A\triangle B)\triangle C=A\triangle(B\triangle C)$

(2) $(A\triangle B)\triangle (B\triangle C)=A\triangle C$

(3) $A\triangle B=C\Leftrightarrow A=B\triangle C$
\end{prob}
